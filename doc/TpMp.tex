% arara: pdflatex
% arara: bibtex
% arara: pdflatex
% arara: pdflatex

\documentclass[a4paper,11pt]{article}
\pdfoutput=1 % if your are submitting a pdflatex (i.e. if you have
             % images in pdf, png or jpg format)

\usepackage[a4paper,
total={170mm,257mm},
left=20mm,
top=20mm,]{geometry}

\usepackage[dvipsnames]{xcolor}
\usepackage{longtable}
\usepackage{authblk}
\usepackage{hyperref}

\usepackage{titlesec}
\titleformat{\section}[block]{\color{blue}\Large\bfseries\filcenter}{}{1em}{}
\titleformat{\subsection}[hang]{\bfseries}{}{1em}{}


\title{\textcolor{red}{\texttt{\Huge TpMp}} - topology mapping}


\author{Andrey Pikelner}
\affil{II.~Institut f\"ur Theoretische Physik, Universit\"at Hamburg,\\
Luruper Chaussee 149, 22761 Hamburg, Germany}


% \emailAdd{andrey.pikelner@desy.de}


% Mathematica commands in blue
\newcommand{\mma}[1]{\textcolor{BlueGreen}{\texttt{#1}}}
% Color tt
\newcommand{\ctt}[1]{\textcolor{OliveGreen}{\texttt{#1}}}

\setcounter{secnumdepth}{0}
\begin{document} 
\maketitle

\abstract{We present program for diagram generation and mapping on
  topology working with \texttt{Mathematica} interface. Available from
  \url{https://github.com/apik/TpMp}}


\section*{Design ideas}

\subsection*{QGRAF input conventions}
\begin{itemize}
\item We use all particles as internal in QGRAF and having momentum
  $p_1,\dots,p_n$. Field out kept blank.
\begin{verbatim}
   in  = q, Q ;
   out = ;
\end{verbatim}

\item All fields in model file must contain field \ctt{type} with
  possible values \ctt{F,M,S,C,A} in addition to
  comutativity flag $\pm$.
\begin{verbatim}
   [ phi, phi, + ; type='S']
\end{verbatim}

\item We set internal momentum to be \ctt{k}
\end{itemize}


\subsection*{DB storage}
\label{sec:dbstore}


\section*{Main commands available in package }

\subsection*{Loading diagrams}
\begingroup
\renewcommand\arraystretch{2}
\begin{longtable}{p{4cm}|p{11cm}}
  \mma{LoadQGRAF["qlist"]}
  & Load \texttt{QGRAF}\cite{Nogueira:1991ex} output \texttt{qlist.yaml} in YAML format produced with the help of
    yaml.sty style file and produce SQLite 3 DB with generated
    diagrams and name \texttt{qlist.sqlite3}.\\
  \mma{nDB=LoadDB["db"]}
  & Load SQLite 3 DB from file \texttt{db.sqlite3} with diagrams and
    return numeric descriptor \texttt{nDB} of open DB.\\
  \mma{GetDia[nDia,nDB]}
  & Retrive diagram with number \texttt{nDia} from DB with descriptor
    \texttt{nDB}.\\
\end{longtable}
\endgroup

When DB loaded all commands need DB decriptor to be specified. If DB
descriptor not specified explicitly - all commands applyed to first
open DB.

\subsection*{Selecting diagrams}

\begingroup
\renewcommand\arraystretch{2}
\begin{longtable}{p{4cm}|p{11cm}}
  \mma{WithField["f"]}
  & Select diagrams with field named "f".\\
  \mma{WithFieldType[T]}
  & Select diagrams with field type \texttt{T}, where \texttt{T} is
  one of: \texttt{F} - for Dirac fermion, \texttt{M} - for Majorana
  fermion, \texttt{S} - for scalar, \texttt{C} - for ghost and
  \texttt{A} - for auxiliary field type.\\
\end{longtable}
\endgroup

\subsection*{Plotting diagrams}


\section{Mapping}
\subsection{\boldmath $U$ and $F$ polynomials}

If we define $P_i=-q_j^2+m_j^2$, than:
\begin{equation}
  \label{eq:MQJdefMass}
  \sum\limits_{j=1}^{n}x_j(-q_j^2+m_j^2)=-\sum\limits_{r=1}^{l}\sum\limits_{s=1}^{l}k_rM_{rs}k_s+\sum\limits_{r=1}^{l}2k_r\cdot
  Q_r+J
\end{equation}
Term $J$ contain all dependence on external momentum products and
masses and can be decomposed:

\begin{equation}
  \label{eq:Jdecompose}
  J=J^{p}+J^{m} = J^{p}(p_i^2,p_i\cdot p_j)+\sum\limits_{j=1}^{n}x_jm_j^2
\end{equation}

If only information about momentum distribution needed we can neglect
masses and use $P$ polynomial instead of $F$

\begin{equation}
  \label{eq:MQJdefNoMass}
  -\sum\limits_{j=1}^{n}x_jq_j^2=-\sum\limits_{r=1}^{l}\sum\limits_{s=1}^{l}k_rM_{rs}k_s+\sum\limits_{r=1}^{l}2k_r\cdot
  Q_r+J^{p}
\end{equation}

And Feynman graph polymnomials defined as
\begin{equation}
  \label{eq:UFMdef}
  U=\det(M),\quad P=\det(M)(J^p+QM^{-1}Q),\quad F=\det(M)(J+QM^{-1}Q).
\end{equation}

$U$ polynomial determine internal structure of diagram, $P$ polynomial
determines external momenta distribution and kinematic of the diagram
and $F$-polynomial determines how masses are distributed between lines.


\begin{verbatim}
<|-x[1] -> 
        <|{} -> <|{1} -> {{{}, T12[{1, k1, m1}]}}|>|>, 

-x[1] - x[2] -> 
        <|{{1, 1}} -> 
             <|{1,  1} -> {{{{1, 1} -> -p1^2}, T12[{1, k1,      m1}, {2, k1 + p1, m2}]}}, 
               {0,  1} -> {{{{1, 1} -> -p1^2}, T02[{2, k1 + p1, m2}, {1, k1, 0}]}}, 
               {1,  0} -> {{{{1, 1} -> -p1^2}, T10[{1, k1,      m1}, {2, k1 + p1, 0}]}}, 
               {0,  0} -> {{{{1, 1} -> -p1^2}, T00[{1, k1,       0}, {2, k1 + p1, 0}]}}|>|>|>
\end{verbatim}

\subsection{External momentum flow}
\label{sec:extflow}

It is possible to map diagram with $p_1,\dots,p_n$ external momentum on topology
with $p_1,\dots,p_k,k<n$ external momenta. This mean that momenta flow
for $p_{k+1},\dots,p_n$ kept according to flow in original
diagram. Momentum to be kept should be marked as optional parameter of
\ctt{MapOnAux} function as \ctt{SplitMomenta->\{p2,p3,...\}}.

\subsection{Internal structure}

For set of diagrams it is possible to define ``projections'' i.e.
diagrams with contraints on kinematical variables: masses and external
momentums.

Projections created with the help of command \texttt{Project["pr",subs]} where
parameters \texttt{"pr"} - unique name of projection and \texttt{subs}
- substitution rules.

% \appendix
% \section{Additions}
% \acknowledgments

\bibliographystyle{plain}
\bibliography{TpMp} 

\end{document}

%%% Local Variables:
%%% mode: latex
%%% TeX-master: t
%%% End:
